
	\begin{frame}
		\frametitle{Pre-requirements}

		\begin{itemize}
			\item[$\bullet$] a little of bash-fu
			\item[$\bullet$] a computer with docker installed (\href{https://docs.docker.com/v17.09/engine/installation/}{official documentation})
		\end{itemize}
		
	\end{frame}
	
	\begin{frame}
		\frametitle{Terminals}
		
		During the workshop, we will need to use several shell at the same time.
		
		\medskip
		
		We need a tool for that, like for example:
		\begin{itemize}
			\item[$\bullet$] tmux (in command line)
			\item[$\bullet$] screen
			\item[$\bullet$] Terminator
			\item[$\bullet$] ... or just open several terminals
		\end{itemize}
	\end{frame}
	
	\begin{frame}
		\frametitle{Terminals}
		
		In this workshop we will use up to four terminals at once.
		
		\bigskip
		
		To keep things clear, these terminals will be named:
		\begin{itemize}
			\item[$\bullet$] Command terminal
			\item[$\bullet$] Monitor terminal
			\item[$\bullet$] Logs terminal
			\item[$\bullet$] Second command terminal
		\end{itemize}
		
		\bigskip
		
		If the commands need to be run in a specific folder, it will be indicated as well.
	\end{frame}

	\begin{frame}
		\frametitle{Folder tree}
		
		For the following, all path will be assume to be under a \textit{training} folder.
		
		\bigskip
		
		For example, assuming the path of \textit{training} is \textit{/training},
		
		the mention \textit{folder/scrip.sh} 
		
		will be equivalent to \textit{/training/folder/script.sh}.
		
	\end{frame}