\section{First worker}

\subsection{Objectives}	
	\begin{frame}
		\frametitle{Objective}
		
		\begin{itemize}
			\item[$\bullet$] Our first task is to create a wonderfull worker that will keep telling us that it is alive.
			\item[$\bullet$] To keep simple, this worker will be in batch (but any langage will work).	
			\item[$\bullet$] Then we will run this worker in a container on our local docker installation.
		\end{itemize}
		
	\end{frame}

\subsection{Create the worker}
	
	\begin{frame}[fragile]
		\frametitle{Our worker code}

		First we place ourself in a working folder WORKER\_DIR. And we create the worker core:
		
		\begin{block}{worker.sh}
			\begin{verbatim}
				while true;do
				  echo "I'm alive and it is $(date)!"
				  sleep 2
				done
			\end{verbatim}
		\end{block}
		
	\end{frame}

	\begin{frame}[fragile]
		\frametitle{Test our worker}
		
		\begin{block}{In a shell}
			\begin{verbatim}
				chmod u+x worker.sh
				./worker.sh
			\end{verbatim}
		\end{block}
		We can kill our worker with hitting \textbf{Ctrl+C}.
		
	\end{frame}
	
\subsection{Work with docker}

	\begin{frame}[fragile]
		\frametitle{Embedded our worker in a container}
		
		We need to write a Dockerfile:
		
		\begin{block}{Dockerfile}
			\begin{verbatim}
				FROM alpine

				COPY worker.sh .

				CMD /worker.sh
			\end{verbatim}
		\end{block}
	\end{frame}
	
	\begin{frame}[fragile]
		\frametitle{Create our image}
		
		We create the image of our worker:
		
		\begin{block}{Command line}
			\begin{verbatim}
				docker build -t tinkou/worker:v1 .
			\end{verbatim}
		\end{block}
		
		This image can be seen locally in docker:

		\begin{block}{Command line}
			\begin{verbatim}
				docker images
			\end{verbatim}
		\end{block}
		
	\end{frame}
	
	\begin{frame}[fragile]
		\frametitle{Run our container}
		
		Now that we have our worker image, it is time to run it in docker:
		
		\begin{block}{Command line}
			\begin{verbatim}
				docker run tinkou/worker:v1
			\end{verbatim}
		\end{block}
		The output of our worker can be seen.
		
		We can kill our container with \textbf{Ctrl+C}.
		
	\end{frame}
	
	\begin{frame}[fragile]
		\frametitle{Time to play a little with our container}
		
		We can start our container in background and with giving it a name:
		\begin{block}{Command line}
			\begin{verbatim}
				docker run --name worker -d tinkou/worker:v1
			\end{verbatim}
		\end{block}
		
		\bigskip
		We can still take a look to the logs:
		\begin{block}{Command line}
			\begin{verbatim}
				docker logs -f worker
			\end{verbatim}
		\end{block}
		We can quit the follow with \textbf{Ctrl+C}.

	\end{frame}
	
	\begin{frame}[fragile]
		\frametitle{Time to play a little with our container}
		
		We can start, stop and see the logs of our container at will:
		\begin{block}{Command line}
			\begin{small}
				\begin{verbatim}
					docker ps
					docker stop worker
					docker ps
					docker start worker
					docker ps
				\end{verbatim}
			\end{small}
		\end{block}
		
		\bigskip
		This is always the same container running. We can see all existing containers to check it:
		\begin{block}{Command line}
			\begin{verbatim}
				docker ps -a
			\end{verbatim}
		\end{block}

	\end{frame}
	
\subsection{Cleaning}	
	
	\begin{frame}[fragile]
		\frametitle{Clean after work}
		
		We stop and remove the containers and then remove the images:
		\begin{block}{Command line}
			\begin{verbatim}
				docker stop $(docker ps -q)
				docker rm $(docker ps -aq)
				docker rmi $(docker images -q)
			\end{verbatim}
		\end{block}
				
		\bigskip
		We can check that nothing remain with:
		\begin{block}{Command line}
			\begin{verbatim}
				docker ps
				docker ps -a
				docker images
			\end{verbatim}
		\end{block}

	\end{frame}
	
\subsection{Going farther}

	\begin{frame}[fragile]
		\frametitle{To go farther...}
		
		To learn more thing about docker, you can check \href{https://docs.docker.com}{Docker official documentation} or use the docker built-in help:
		
		\begin{block}{In a shell}
			\begin{verbatim}
				docker help
				docker help build
				...
			\end{verbatim}
		\end{block}
	\end{frame}