\section{Managing different configurations}

\subsection{Objectives}

	\begin{frame}
		\frametitle{Objective}
		
		Now that we have a worker in kubernetes, we wants to be able to configure it.
		
		Also, we want to be able to differentiate between development and production configuration.
		
	\end{frame}
	
\subsection{Use kustomize}	
	
	\begin{frame}
		\frametitle{Kustomize among other solutions}
		
		\begin{center}
		\includegraphics[height=7.5cm]{../../../resources/color/choiceConfigKind.pdf}
		\end{center}
	\end{frame}
	
	\begin{frame}
		\frametitle{Layer based configuration}
		
		Kustomize is a tool now integrated in kubectl.
		
		\bigskip
		It manage the configuration using a layer based system.
		
		That means that a configuration is the result of a base configuration, on which layers are applied.
		
		Each layer can contains new resources or new patches.
		
		\bigskip
		As a base layer is considered as a resource, that enable to define a configuration as the concatenation of several other configurations and their patches.
	\end{frame}
	
	\begin{frame}[fragile]
		\frametitle{Initialize the base}
		
		We are creating a folder tree to sort our files
		\begin{block}{Command line 1 in folder worker}
			\begin{verbatim}
				mkdir kube
				cd kube
				mkdir base
				cd base
			\end{verbatim}
		\end{block}
		
		\bigskip
		
		\begin{footnotesize}
			Command line 2 still had "\verb!stern worker!" running
			
			Command line 3 still had "\verb!kubectl get pods -w!" running
		\end{footnotesize}
	\end{frame}
	
	\begin{frame}[fragile]
		\frametitle{Initialize the base}
		
		We are starting with a source.yaml file describing our deployment
		\begin{block}{source.yaml}
			\begin{tiny}
				\begin{verbatim}
					apiVersion: apps/v1
					kind: Deployment
					metadata:
					  name: worker
					  labels:
					    tinkou: worker
					spec:
					  selector:
					    matchLabels:
					      tinkou: worker
					  template:
					    metadata:
					      labels:
					        tinkou: worker
					    spec:
					      containers:
					      - name: worker
					        image: tinkou/worker:v1
				\end{verbatim}
			\end{tiny}
		\end{block}
	\end{frame}
	
	\begin{frame}[fragile]
		\frametitle{Initialize the base}
		\begin{block}{Command line 1 in folder base}
			\begin{verbatim}
				touch kustomization.yaml
				kustomize edit fix
				kustomize edit add resource source.yaml
				kustomize build
				kubectl apply -k .
			\end{verbatim}
		\end{block}
	\end{frame}
	
	\begin{frame}[fragile]
		\frametitle{Initialize the base}
		As we modify an immutable field, we need to delete the previous deployment before:
		\begin{block}{Command line 1 in folder base}
			\begin{verbatim}
				kubectl delete deployment worker
				kubectl apply -k .
			\end{verbatim}
		\end{block}
	\end{frame}
	
	\begin{frame}[fragile]
		\frametitle{Add a parameter to our worker}
		
		Modify the worker to use a parameter:
		\begin{block}{worker.sh}
			\begin{verbatim}
				while true; do
				  echo "I'm $NAME and it is $(date)"
				  sleep 2
				done
			\end{verbatim}
		\end{block}
	\end{frame}
	
	\begin{frame}[fragile]
		\frametitle{Deploy the new version in our cluster}
		
		First we need to create a new version of the image:
		\begin{block}{Command line 1 in folder worker}
			\begin{verbatim}
				docker build -t tinkou/worker:v2 .
			\end{verbatim}
		\end{block}
	\end{frame}
	
	\begin{frame}[fragile]
		\frametitle{Deploy the new version in our cluster}
		
		Adapt in source.yaml the field \verb!.spec.template.spec!:
		\begin{block}{source.yaml}
			\begin{tiny}
				\begin{verbatim}
					apiVersion: apps/v1
					kind: Deployment
					metadata:
					  name: worker
					  labels:
					    tinkou: worker
					spec:
					  selector:
					    matchLabels:
					      tinkou: worker
					  template:
					    metadata:
					      labels:
					        tinkou: worker
					    spec:
					      containers:
					      - name: worker
					        image: tinkou/worker:v2
					        env:
					        - name: NAME
					          value: Bink
				\end{verbatim}
			\end{tiny}
		\end{block}
	\end{frame}
	
	\begin{frame}[fragile]
		\frametitle{Deploy the new version in our cluster}
			
		And finally apply the modification:
		\begin{block}{Command line 1 in folder worker}
			\begin{small}
					\begin{verbatim}
					kubectl apply -k kube/base
				\end{verbatim}
			\end{small}
		\end{block}
		
		\bigskip
		
		\visible<2->{
			To clean what have been deployed:
		}
		\begin{block}{Command line 1 in folder worker}<2->
			\begin{verbatim}
				kubectl delete -k kube/base
			\end{verbatim}
		\end{block}
	\end{frame}
	
	\begin{frame}
		\frametitle{Deploy the new version in our cluster}
		
		There are too many operations.
		
		\bigskip
		Is there a way to simplify this?
	\end{frame}
	
\subsection{Use skaffold}	
	
	\begin{frame}
		\frametitle{Skaffold}
		
		Skaffold is a developer oriented tool create to simplify the packaging and deployment on kubernetes.
		
		\medskip
		The Skaffold documentation can be found \href{https://github.com/GoogleContainerTools/skaffold}{her}.

	\end{frame}
	
	\begin{frame}[fragile]
		\frametitle{Initialize skaffold}
		
		Let's initialize skaffold:
		\begin{block}{Command line 1 in folder worker}
			\begin{verbatim}
				skaffold init
			\end{verbatim}
			Follow the application command line interface.
		\end{block}
	\end{frame}
	
	\begin{frame}[fragile]
		\frametitle{Initialize skaffold}

		Skaffold detect the yaml and skaffold.yaml needs to be configured to use kustomize:
		\begin{block}{skaffold.yaml}
			\begin{footnotesize}
				\begin{verbatim}
					apiVersion: skaffold/v1beta15
					kind: Config
					metadata:
					  name: worker
					build:
					  artifacts:
					  - image: tinkou/worker
					deploy:
					  kustomize:
					    path: kube/base
				\end{verbatim}						
			\end{footnotesize}
		\end{block}
		The \verb!apiVersion! can change with skaffold version.
	\end{frame}
	
	\begin{frame}[fragile]
		\frametitle{Using Skaffold}
		
		Let's use Skaffold to build our image:
		\begin{block}{Command line 1 in folder worker}
			\begin{verbatim}
				skaffold run
			\end{verbatim}
		\end{block}
		
		\bigskip
		
		\visible<2->{
			A particularity of Skaffold is that to work with minikube, the current context of kubectl must be "minikube".\\
			The current context can be with:
		}

		\begin{block}{Command line 1}<2->
			\begin{verbatim}
				kubectl config current-context
			\end{verbatim}
		\end{block}				
	\end{frame}
	
	\begin{frame}[fragile]
		\frametitle{Using skaffold}
		
		We are going to test a little skaffold commands:
		\begin{block}{Command line 1 in folder worker}
			\begin{verbatim}
				skaffold build
				skaffold run
				skaffold delete
			\end{verbatim}					
		\end{block}
		
		\begin{description}
			\item[build] only build the image and push it in the registry (except in minikube)
			\item[run] build, push and run the image in kubectl current cluster
			\item[delete] delete run element from the kubectl current cluster
		\end{description}
	
	\end{frame}
	
	\begin{frame}[fragile]
		\frametitle{Using skaffold}
		
		Lets try the dev function:
		\begin{block}{Command line 1 in folder worker}
			\begin{verbatim}
				skaffold dev
			\end{verbatim}					
		\end{block}

		It works like run, but displaying logs and keeping the hand on the shell.
	\end{frame}
	
	\begin{frame}
		\frametitle{Using skaffold}

		Open a command line in a new window 4.		
		
		Try to modify the file worker.sh in this new command line.
		
		\bigskip
		
		And look at how skaffold dynamically build and deploy each time the file is saved.
	\end{frame}
	
	\begin{frame}[fragile]
		\frametitle{Isolate the configuration in a configmap}

		Add a file conf.env in the folder base:
		\begin{block}{conf.env}
			\begin{verbatim}
				NAME=Trent
			\end{verbatim}
		\end{block}				
		
		\begin{block}{Command line 4 in the folder kube/base}
			\begin{verbatim}
				kustomize edit add configmap worker \
				                   --from-env-file conf.env
			\end{verbatim}
		\end{block}
	\end{frame}

	\begin{frame}[fragile]
		\frametitle{Isolate the configuration in a configmap}

		\begin{block}{source.yaml}
			\begin{tiny}
				\begin{verbatim}
					apiVersion: apps/v1
					kind: Deployment
					metadata:
					  name: worker
					  labels:
					    tinkou: worker
					spec:
					  selector:
					    matchLabels:
					      tinkou: worker
					  template:
					    metadata:
					      labels:
					        tinkou: worker
					    spec:
					      containers:
					      - name: worker
					        image: tinkou/worker:v2
					        envFrom:
					        - configMapRef:
					            name: worker
				\end{verbatim}
			\end{tiny}			
		\end{block}
	\end{frame}
	
	\begin{frame}[fragile]
		\frametitle{Define a local configuration}
		
		Create a folder \verb!local! in \verb!worker/kube!, and add in it the file \verb!patch.yaml!:
		\begin{block}{patch.yaml}
			\begin{verbatim}
				apiVersion: v1
				kind: ConfigMap
				metadata:
				  name: worker
				data:
				  NAME: Iris
			\end{verbatim}
		\end{block}
		
		\medskip
		
		\visible<2->{
			Nothing happen in skaffold dev.
		}
	\end{frame}
	
	\begin{frame}[fragile]
		\frametitle{Define a local configuration: kustomize}
		
		First create a new kustomize project with a base resource and the patch:
		\begin{block}{Command line 4 in folder worker/kube/local}
			\begin{verbatim}
				touch kustomization.yaml
				kustomize edit fix
				kustomize edit add resource ../base
				kustomize edit add patch patch.yaml
			\end{verbatim}
		\end{block}
		
		\visible{
			Still nothing in skaffold dev.
		}
	\end{frame}
	
	\begin{frame}[fragile]
		\frametitle{Define a local configuration: skaffold}
		
		Indicate to skaffold to use this local values when using minikube:
		\begin{block}{skaffold.yaml}
			\begin{tiny}
				\begin{verbatim}
						apiVersion: skaffold/v1beta15
						kind: Config
						metadata:
						  name: worker
						build:
						  artifacts:
						  - image: tinkou/worker
						deploy:
						  kustomize:
						    path: kube/base
						profiles:
						- name: local
						  activation:
						  - kubeContext: minikube
						  deploy:
						    kustomize:
						      path: /kube/local
				\end{verbatim}
			\end{tiny}
		\end{block}
	
	\end{frame}
	
	\begin{frame}[fragile]
		\frametitle{Skaffold dev modifications detection}
		
		Now the modification is deployed.
		
		Lets modify the base configuration:
		\begin{block}{kube/base/conf.env}
			\begin{verbatim}
				NAME=Dor
			\end{verbatim}
		\end{block}
		
		\visible<2->{
			Skaffold do not react.
			
			\medskip
			
			Try to modify several files to see how skaffold detection works.
		}
	\end{frame}
	
	\begin{frame}[fragile]
		\frametitle{Cleaning}
		
		If you kill the command \verb!skaffold dev! with \textbf{Ctrl+C}, skaffold will clean what it has deployed.
	\end{frame}
	
	\begin{frame}
		\begin{center}
			Questions?
		\end{center}
	\end{frame}