\section{Web service}

\subsection{Objectives}

	\begin{frame}
		\frametitle{Objectives}
		
		Our project now needs a web service.

		\bigskip
		This web service needs to be monitored and scalable.		
		
	\end{frame}
	
\subsection{Creating the web service}
	\begin{frame}[fragile]
		\frametitle{Creating the web service itself}
		
		Create a new working folder \verb!web!.

		\begin{block}{\href{https://github.com/Tinkou/trainings/blob/master/trainings/sources/dev-onboarding/files/web/index.html}{web/index.html}}
			\begin{small}
				\begin{verbatim}
					<!DOCTYPE HTML>
					<html>
					  <head>
					    <meta charset="UTF-8">
					    <title>Web service training</title>
					  </head>
					  <body>
					    <h1>Hello World!</h1>
					  </body>
					</html>
				\end{verbatim}
			\end{small}
		\end{block}			
	\end{frame}
	
	\begin{frame}[fragile]
		\frametitle{Creating the web service image}
		
		\begin{block}{\href{https://github.com/Tinkou/trainings/blob/master/trainings/sources/dev-onboarding/files/web/Dockerfile}{web/Dockerfile}}
			\begin{small}
				\begin{verbatim}
					FROM nginx:alpine
					COPY index.html /usr/share/nginx/html
				\end{verbatim}
			\end{small}
		\end{block}
		
		\begin{block}{Command terminal in folder web}
			\begin{small}
				\begin{verbatim}
					docker build -t tinkou/web:v1 .
				\end{verbatim}
			\end{small}
		\end{block}
		
	\end{frame}
	
	\begin{frame}[fragile]
		\frametitle{Creating kubernetes deployment configuration}
		\begin{block}{\href{https://github.com/Tinkou/trainings/blob/master/trainings/sources/dev-onboarding/files/web/deployment.yaml}{web/deployment.yaml}}
			\begin{tiny}
				\begin{verbatim}
					apiVersion: apps/v1
					kind: Deployment
					metadata:
					  name: web
					  labels:
					    tinkou: web
					spec:
					  selector:
					    matchLabels:
					      tinkou: web
					  template:
					    metadata:
					      labels:
					        tinkou: web
					    spec:
					      containers:
					      - name: service
					        image: tinkou/web:v1
				\end{verbatim}
			\end{tiny}
		\end{block}
	\end{frame}
	
	\begin{frame}[fragile]
		\frametitle{Creating kubernetes deployment configuration}

		\begin{block}{Logs terminal}
			\begin{small}
				\begin{verbatim}
					stern -l tinkou=web
				\end{verbatim}
			\end{small}
		\end{block}
		
		\begin{block}{Monitor terminal}
			\begin{verbatim}
				watch kubectl get all
			\end{verbatim}
		\end{block}
		
		\begin{block}{Command terminal in folder web}
			\begin{verbatim}
				kubectl apply -f deployment.yaml
			\end{verbatim}
		\end{block}
	\end{frame}
	
	\begin{frame}
		\frametitle{Creating kubernetes deployment configuration}
		
		A deployment, a replicaset and a pod have been created.
		
		\bigskip
		
		But how to access this web service?
	\end{frame}
	
\subsection{Accessing a web service}
	\begin{frame}[fragile]
		\frametitle{Exposing a web service}
		
		Pods have an IP…
		\begin{block}{Command terminal}
			\begin{verbatim}
				kubectl get pod -o wide
			\end{verbatim}
		\end{block}
		… but this IP is internal to the cluster.
		\medskip
		\visible<2->{
			\begin{center}
				---
			
				\medskip
				In kubernetes, pods are exposed via services.
			\end{center}
		}
	\end{frame}
	
	\begin{frame}[fragile]
		\frametitle{Exposing a web service}		
		
		\begin{block}{\href{https://github.com/Tinkou/trainings/blob/master/trainings/sources/dev-onboarding/files/web/service.yaml}{web/service.yaml}}
			\begin{footnotesize}
				\begin{verbatim}
					apiVersion: v1
					kind: Service
					metadata:
					  name: my-service
					  labels:
					    tinkou: web
					spec:
					  type: NodePort
					  selector:
					    tinkou: web
					  ports:
					  - port: 80
					    protocol: TCP
					    nodePort: 32001
			\end{verbatim}
			\end{footnotesize}
		\end{block}
	\end{frame}
	
	\begin{frame}[fragile]
		\frametitle{Exposing a web service}
		
		\begin{block}{Command terminal in folder web}
			\begin{verbatim}
				kubectl apply -f service.yaml
				curl http://$(minikube ip):32001
			\end{verbatim}
		\end{block}
		
		\medskip
		
		Now we can access the service.
		
		It is also possible to display it in a web browser (by replacing \verb!$(minikube ip)! by its value).
	\end{frame}
	
	\begin{frame}[fragile]
		\frametitle{How does it work?}
		
		\begin{itemize}
			\item[$\bullet$] the minikube VM has its IP given by \verb!minikube ip!
			\item[$\bullet$] as a NodePort, the service listen to the port 32001 of all nodes
			\item[$\bullet$] the service redirects requests in round robin to pods matching the selector
			\item[$\bullet$] pods listen to the same ports as their containers
			\item[$\bullet$] at last the container can answer to requests
		\end{itemize}
	\end{frame}
	
\subsection{Cleaning up}

	\begin{frame}[fragile]
		\frametitle{Cleaning up}
		
		\begin{block}{Command terminal in folder web}
			\begin{verbatim}
				kubectl delete -f deployment.yaml
				kubectl delete -f service.yaml
			\end{verbatim}
		\end{block}
	\end{frame}
	
	\begin{frame}
		\begin{center}
			Questions?
		\end{center}
	\end{frame}