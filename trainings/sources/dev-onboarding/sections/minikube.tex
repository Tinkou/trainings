\begin{frame}
	\frametitle{In local with minikube}
	
	We need to install the tools used during the workshops:
	\begin{description}[leftmargin=!,labelwidth=\widthof{\bfseries Kustomize}]
		\item[Docker] \href{https://hub.docker.com/search/?type=edition&offering=community}{Official release page}
		\item[Kubectl] \href{https://kubernetes.io/docs/tasks/tools/install-kubectl/}{Official installation documentation}
		\item[Minikube] \href{https://kubernetes.io/docs/tasks/tools/install-minikube/}{Official installation documentation}
		\item[Kustomize] \href{https://github.com/kubernetes-sigs/kustomize/blob/master/docs/INSTALL.md}{Official installation documentation}
		\item[Skaffold] \href{https://skaffold.dev/docs/getting-started/\#installing-skaffold}{Official installation documentation}
		\item[Stern] \href{https://github.com/wercker/stern}{Official installation documentation}
	\end{description}	
\end{frame}

\begin{frame}
	\frametitle{Warning about Kustomize and Skaffold}
	
	As it is now, Kustomize and Skaffold are still under construction. These tools give a great help but the version evolve quickly and the files version can change.
	
	To avoid compatibility issues, there is a need to define the update policy for those tool's version.
	
	\bigskip
	
	\begin{center}
		\textit{As these problematic depends on the organisation of the team or company, it will not be talked here.}
	\end{center}
	
\end{frame}

\begin{frame}[fragile]
	\frametitle{In local with minikube}
	
	Minikube consist of a single node kubernetes cluster. Here a few commands useful for the workshop:
	
	\begin{block}{Get the cluster IP. This IP will be used as the one of each kubernetes servers (master or node):}
		\begin{verbatim}
			minikube ip
		\end{verbatim}
	\end{block}
	
	\medskip
	For the registry, we will use the docker local registry of minikube.
\end{frame}