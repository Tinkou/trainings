\section{Working in Kubernetes}

	\begin{frame}
		\frametitle{How to access to a Kubernetes cluster}
		
		First thing first, to deploy our worker on kubernetes, we need to access to a cluster.
		
		\bigskip
		For that, we need to install kubectl on our worstation ( \href{https://kubernetes.io/docs/tasks/tools/install-kubectl/}{installation documentation}).
		 
		 \bigskip
		The documentation give instruction to configure the cluster access (using a config file given by your sysadmin).

		\bigskip		
		Very usefull, the documentation explain how to enable the completion.
	\end{frame}
	
	\begin{frame}[fragile]
		\frametitle{Working in a dedicated namespace}
		
		To avoid impacting other components, we are going to isolate ourselves in a namespace:
		\begin{block}{Command line}
			\begin{verbatim}
				kubectl get namespaces
				kubectl create namespace my-namespace
				kubectl get ns
			\end{verbatim}
		\end{block}
	\end{frame}
		
	\begin{frame}[fragile]
		\frametitle{Working in a dedicated namespace}
	
		Then we configure kubectl to use his context:
		\begin{block}{Command line}
			\begin{verbatim}
				kubectl config get-context
				
				kubectl config set-context training \
							--cluster=kubernetes \
							--user=kubernetes-admin \
							--namespace=my-namespace
							
				kubectl config use-context training
				kubectl config get-context
			\end{verbatim}
		\end{block}
	\end{frame}
	
	\begin{frame}[fragile]
		\frametitle{Running our first container in the cluster}
	
		To begin, we are just going to deploy a container sending a ping:
		\begin{block}{Command line}
			\begin{verbatim}
				kubectl run pingpong --image=alpine ping 1.1.1.1
				
				kubectl logs -f pingpong-XXXX-XXXX
				kubectl get pods
				kubectl get deployments -o wide
			\end{verbatim}
		\end{block}
	\end{frame}
	
	\begin{frame}[fragile]
		\frametitle{Looking at logs in a kubernetes cluster}
	
		We want to display the logs of this container:
		\begin{block}{Command line in window 1}
			\begin{verbatim}
				kubectl logs -f pingpong-XXXX-XXXX
			\end{verbatim}
		\end{block}
		
		Something happens and we need to recreate the pod:
		\begin{block}{Command line in window 2}
			\begin{verbatim}
				kubectl get pods -w
			\end{verbatim}
		\end{block}
		\begin{block}{Command line in window 3}
			\begin{verbatim}
				kubectl delete pod pingpong-XXXX-XXXX
			\end{verbatim}
		\end{block}
	\end{frame}
	
	\begin{frame}[fragile]
		\frametitle{Looking at logs in a kubernetes cluster}
	
		There are several problems:
		\begin{itemize}
			\item we do not see which logs comes from which pod
			\item in fact we do not see the new containers logs
			\item the follow is broken by the operation
		\end{itemize}
		
		How to solve this and be able to follow logs even if pods are moving?
	\end{frame}

	\begin{frame}[fragile]
		\frametitle{Using stern to look at logs}

		Stern installation instructions can be found \href{https://github.com/wercker/stern}{here}.
		
		\bigskip
		So, we are going the redo the operation but this time using stern:
		\begin{block}{Command line in window 1}
			\begin{verbatim}
				stern pingpong
			\end{verbatim}
		\end{block}
				\begin{block}{Command line in window 2}
			\begin{verbatim}
				watch kubectl get pods
			\end{verbatim}
		\end{block}
		\begin{block}{Command line in window 3}
			\begin{verbatim}
				kubectl delete pod pingpong-XXXX-XXXX
			\end{verbatim}
		\end{block}
	\end{frame}
	
	\begin{frame}[fragile]
		\frametitle{Clean the namespace}
		
		Before returning to our worker, a little cleaning can be wise:
		\begin{block}{Command line1}
			\begin{verbatim}
				kubectl delete deployment pinpong
				kubectl get all
			\end{verbatim}
		\end{block}
	\end{frame}
	
	\begin{frame}[fragile]
		\frametitle{Running our worker in kubernetes}
		
		Now that we know how to run something in kubernetes, we are going to do it with our worker:
		\begin{block}{Command line in window 2}
			\begin{verbatim}
				stern worker
			\end{verbatim}
		\end{block}		
		\begin{block}{Command line in window 3}
			\begin{verbatim}
				kubectl get pods -w
			\end{verbatim}
		\end{block}
		\begin{block}{Command line in window 1}
			\begin{verbatim}
				docker images
				docker build -t tinkou/worker:v1 .
				kubectl run worker --image=tinkou/worker:v1
			\end{verbatim}
		\end{block}
	\end{frame}
	
	\begin{frame}[fragile]
		\frametitle{Troubleshooting}
		
		Kubectl get pods indicate that something went wrong.
		
		Check kubernetes object to find the problem root cause:
		\begin{block}{Command line in window 1}
			\begin{verbatim}
				kubectl	describe deployment worker
				kubectl describe replicaset worker-XXXX
				kubectl describe pod worker-XXXX-XXXX
			\end{verbatim}
		\end{block}
		
	\end{frame}
	
	\begin{frame}[fragile]
		\frametitle{Using a registry}
		
		As our cluster can't access our local image, let's push it in a repository instead:
		\begin{block}{Command line 1}
			\begin{verbatim}
				docker tag tinkou/worker:v1 <reg>/training/worker:v1
				docker login <reg>
				docker push <reg>/training/worker-<id>:v1
			\end{verbatim}
		\end{block}
		\verb|<reg>| is the registry given by your sysadmin.
		\verb|<id>| is used to differenciate the images between the different trainees.

	\end{frame}
	
	\begin{frame}
		\frametitle{Using a registry}
	
		And we try again by forcing a pod reconstruction::
		\begin{block}{Command line 1}
			kubectl edit deployment worker
			
			\bigskip
			And change the image field value.
		\end{block}
	
	\end{frame}

	\begin{frame}
		\frametitle{Adding docker registry credentials}
		
		The logs on the pod indicate that docker registry credentials aren't valid.
		
		To facilitate the training, we can add to our namespace the credentials already defined in default in the secret regcred:
		\begin{block}{Command line 1}
				kubectl -n default get secret regcred -o yaml >regcred.yaml
				
				\medskip
				vi regcred.yaml
				
				remove field namespace
				
				\medskip
				kubectl apply -f regcred.yaml
		\end{block}
	\end{frame}
	
	\begin{frame}
		\frametitle{Adding docker registry credentials}
		
		Get our worker deployment yaml and add it the registry credentials:
		\begin{block}{deployment.yaml}
			---
			
				imagePullSecrets:
				- name: regcred
				
			---
			
			kubectl apply -f deployment.yaml
		\end{block}
	\end{frame}


%	- docker build
%	- docker push
%	- kubectl create deployment —image=trucmuche
%	- kubectl get pods
%	- kubectl logs sur le pod